\chapter*{Úvod}
\phantomsection
\addcontentsline{toc}{chapter}{Úvod}

\indent Cílem diplomové práce je navrhnout zařízení,
které bude schopno ověřit správnost interního zapojení In Circuit Testing (ICT) testerů.
ICT tester slouží k odhalování výrobních chyb a programování Desek Plošných Spojů (DSP - dále
označováno jako PCB z anglického Printed Circuit Board)
pomocí připojení velkého množství testovacích jehel. Testovací jehly jsou připojovány do určených
míst na PCB, kde se následně provádí různé testy.\par

Diplomová práce je vypracována ve spolupráci s firmou Čevor inovation a.s.,
která se zabývá výrobou ICT testerů. Při výrobě těchto testerů vzniká potřeba kontroly správnosti zapojení fixture části.
Fixture část (Obr. \ref{fig:ICT_tester}) je odlišná pro každé PCB a obsahuje řádově tisíce propojení.
Zařízení navrhované v této práci tak není ICT testerem samotným,
ale slouží právě k ověření správnosti interního zapojení fixture části ICT testeru.
Jedná se tedy o tester ICT testerů.\par

Ověření správnosti propojení se určuje odporem elektrické cesty.
Zařízení tedy musí být schopno měřit odpor mezi kterýmikoliv dvěma testovanými body.
Zároveň jsou zde kladeny požadavky na škálovatelnost celého řešení z pohledu počtu testovaných propojení.
V diplomové práci se počítá s návrhem zařízení, které má 4000 testovaných bodů.
Fyzické (reálné) propojení testeru je následně porovnáno s výrobními daty ICT testeru.
Výsledky testu jsou přenášeny do PC pomocí Ethernetu,
kde jsou následně zpracovány. Výsledkem testu by pak měla být jasná identifikace chyb.\par

Diplomová práce popisuje funkčnost, hardwarové řešení testeru (návrh PCB) a softwarové řešení v
podobě firmwaru pro PCB a uživatelského programu pro PC. 